%%%%%%%%%%%%%%%%%%%%%%%%%%%%%%%%%%%%%%%%%%%%%%%%
% BOZZA PER TESI DI LAUREA IN LATEX
%%%%%%%%%%%%%%%%%%%%%%%%%%%%%%%%%%%%%%%%%%%%%%%%

%\documentclass[12pt,titlepage]{book}
\documentclass[a4paper,12pt]{article}
\usepackage[english]{babel}
%\usepackage{graphics}
%\usepackage{url,amsfonts,epsfig}
%\usepackage[applemac]{inputenc} %comando per le lettere accentate se usate mac  
\usepackage[utf8]{inputenc} % comando per le lettere accentate se usate pc  
\usepackage[nottoc]{tocbibind}
\usepackage{hyperref}

\hypersetup{
colorlinks=true,
linkcolor=blue
}

\title{\textsc{Simulation of HL-LHC inner tracker performances}}
\author{Federico Massa}

\begin{document}

%%%% Opzione per interlinea 2
%%%\baselineskip 18pt

\maketitle

\tableofcontents
%%\listoffigures
%%\listoftables
\newpage

\section{Introduction} \label{Introduction}

dfdfdffsjdfl

\newpage

\section{Detector} \label{Detector}

This section briefly describes the current experimental apparatus and the reasons as to why it cannot withstand the conditions of HL-LHC.
ATLAS detector is currently composed by 
\begin{itemize}
\item a \textbf{magnet(ic?) system}
\item an \textbf{inner tracker}
\item an \textbf{electromagnetic calorimeter}
\item an \textbf{hadronic calorimeter}
\item a \textbf{muon spectrometer}
\item a \textbf{trigger system}
\item a \textbf{data acquisition system (DAQ)}
\item an \textbf{offline system}
\end{itemize}

\subsection{The magnet(ic?) system}
\cite{magnetic_system}


\newpage
\section{Conclusions}
\label{Conclusions}
\baselineskip 25pt
\baselineskip 5pt
\baselineskip 16pt

%%% EVENTUALE
\appendix

%%% OBBLIGATORIA:

\bibliographystyle{unsrt}
\bibliography{biblio}

\end{document}
